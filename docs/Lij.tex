\documentclass[10pt,a4paper,notitlepage]{article}
\usepackage{fontspec}
\defaultfontfeatures{Mapping=tex-text}
\usepackage{xunicode}
\usepackage{xltxtra}
%\setmainfont{???}
\usepackage{amsmath}
\usepackage{amsfonts}
\usepackage{amssymb}
\usepackage{graphicx}
\usepackage{cleveref}
\usepackage{subfigure}
\usepackage{a4wide}
\usepackage{siunitx}
\usepackage[textsize=scriptsize,colorinlistoftodos,backgroundcolor=blue!20,textwidth=3.7cm]{todonotes}
\usepackage{booktabs}
\usepackage{rotating} 
\usepackage{adjustbox}
\usepackage{mhchem}

\newcommand{\degs}{\,^{\circ}}
\providecommand{\abs}[1]{\lvert#1\rvert}
\newcommand{\vardiff}[2]{\frac{\delta#1}{\delta#2}}
\newcommand{\partdiff}[2]{\frac{\partial#1}{\partial#2}}
\providecommand{\diff}[2]{\frac{\mathrm{d}#1}{\mathrm{d}#2}}
\providecommand{\secdiff}[2]{\frac{\mathrm{d}^2#1}{\mathrm{d}#2^2}}
\providecommand{\vbf}[1]{\boldsymbol{#1}}
\providecommand{\gradient}[0]{\vbf{\nabla}}
\newcommand{\unit}[1]{\ensuremath{\, \mathrm{#1}}}
\newcommand{\code}[1]{\texttt{#1}}
%\DeclareMathOperator{\Tr}{Tr}
\newcommand{\norm}[1]{\left\lVert#1\right\rVert}
\providecommand{\inte}[4]{\int_{#1}^{#2}{#3}\,\mathrm{d}{#4}}
\providecommand{\nablat}{\widetilde{\gradient}}
\providecommand{\tr}[1]{\mathrm{tr}\left(#1\right)}

\begin{document}

Equation (25) in Moelans paper \cite{moelans2011quantitative} defined the mobility as
\begin{equation}
  L_{\alpha,\beta} = \frac{\sqrt{2}mg(\gamma)}{\kappa I_\phi(\gamma)\zeta_{\alpha,\beta}},
\end{equation}
using $\gamma=1.5$ leads to $g(\gamma)=\sqrt{2}/3$ and $I_\phi(\gamma)=1/2$.
Inserting this gives
\begin{equation}
  L_{\alpha,\beta} = \frac{4m}{3\kappa\zeta_{\alpha,\beta}},
\end{equation}
where $m$ and $\kappa$ are constants that depends on the surface energy and the width of the interface region which I assume are the same for all interfaces.
The $\zeta_{\alpha,\beta}$ is taken as
\begin{equation}
  \zeta_{\alpha,\beta} = \frac{(x_\alpha^{eq}-x_\beta^{eq})^2}{0.5(M_\alpha+M_\beta)},
\end{equation}
with $x_\alpha^{eq}$ and $x_\beta^{eq}$ being the equilibrium composition of Sn in the $\alpha$ and $\beta$ phases and
\begin{equation}
  M_\alpha = \frac{D_\alpha}{\secdiff{f_\alpha}{x_\alpha}}, \quad   M_\beta = \frac{D_\beta}{\secdiff{f_\beta}{x_\beta}}
\end{equation}
are the diffusion mobilities.

Using the parabolic approximation of the free energy $f_\rho=\frac{A_\rho}{2}(x_\rho-x_{\rho,0})^2+C_\rho$ the diffusion mobility becomes
\begin{equation}
  M_\rho = \frac{D_\rho}{A_\rho}.
\end{equation}

For the numerical examples in the paper the diffusion coefficients are taken as $D_{\ce{Cu}}=\num{1e-25}$, $D_{\ce{Cu6Sn5}}=\num{1e-16}$ and $D_{\ce{Sn}}=\SI{1e-14}{\square\meter\per\second}$.
The energy coefficients are $A_{\ce{Cu}} =\num{1e8}$, $A_{\ce{Cu6Sn5}}=\num{1e9}$, and $A_{\ce{Sn}}=\SI{1e9}{\joule\per\cubic\meter}$
The huge difference in diffusion coefficients is what creates the difference in mobilities.
For the interface between \ce{Cu} and \ce{Cu6Sn5} I get $\zeta_{\ce{Cu},\ce{Cu6Sn5}}=\num{3.23e24}$ and for the interface between \ce{Cu6Sn5} and \ce{Sn} I get $\zeta_{\ce{Cu6Sn5},\ce{Sn}}=\num{5.42e22}$.
This leads to the mobility of the interface between \ce{Cu6Sn5} and \ce{Sn} being about 500 times higher than the mobility between \ce{Cu} and \ce{Cu6Sn5}.

In the simulation I use \si{\electronvolt} and \si{\nano\meter} instead of \si{\joule} and \si{\meter} so the numbers are a bit nicer but their ratio is the same.
\bibliographystyle{plain}
\bibliography{ref}
\end{document}