\documentclass[12pt,a4paper]{article}
\usepackage{fontspec}
\defaultfontfeatures{Mapping=tex-text}
\usepackage{xunicode}
\usepackage{xltxtra}
%\setmainfont{???}
\usepackage{polyglossia}
\setdefaultlanguage{english}
\usepackage{amsmath}
\usepackage{amsfonts}
\usepackage{amssymb}
\usepackage{color}
\usepackage{soul}
\usepackage{hyperref}
\usepackage[capitalise]{cleveref}
\usepackage{showlabels}
\usepackage[version=3]{mhchem}
\usepackage{pdfpages}
\usepackage{siunitx}
\usepackage{cancel}
\usepackage{booktabs}
\usepackage{tablefootnote}
\usepackage{todonotes}
\usepackage{physics}

\newcommand{\degs}{\,^{\circ}}
\providecommand{\abs}[1]{\lvert#1\rvert}
\newcommand{\vardiff}[2]{\frac{\delta#1}{\delta#2}}
\newcommand{\partdiff}[2]{\frac{\partial#1}{\partial#2}}
\providecommand{\diff}[2]{\frac{\mathrm{d}#1}{\mathrm{d}#2}}
\providecommand{\secdiff}[2]{\frac{\mathrm{d}^2#1}{\mathrm{d}#2^2}}
\providecommand{\vbf}[1]{\boldsymbol{#1}}
\providecommand{\gradient}[0]{\vbf{\nabla}}
\newcommand{\unit}[1]{\ensuremath{\, \mathrm{#1}}}
\newcommand{\code}[1]{\texttt{#1}}
%\DeclareMathOperator{\Tr}{Tr}
\providecommand{\norm}[1]{\left\lVert#1\right\rVert}
\providecommand{\inte}[4]{\int_{#1}^{#2}{#3}\,\mathrm{d}{#4}}
\providecommand{\nablat}{\widetilde{\gradient}}
\providecommand{\tr}[1]{\mathrm{tr}\left(#1\right)}

\providecommand{\vint}[2]{\left(#1,#2\right)}
\providecommand{\sint}[2]{\left<#1,#2\right>}

\begin{document}

\title{Multiphase KKS model with crystal plasticity in Moose}
\author{Johan Hektor}
\maketitle

\section{Introduction}
This document contain notes on the implementation of the phase field model presented in \cite{hektor2016coupled} in Moose.
The following special notation of volume and surface integrals will be used, in accordance with the official Moose documentation;
\begin{equation}
  \inte{}{}{a\cdot b}{V}=\vint{a}{b}, \qquad \inte{}{}{a\cdot b}{S}=\sint{a}{b}
\end{equation}
The sets of phase field variables and phase compositions are denoted $\vbf{\eta}$ and $\vbf{x}$, respectively. 
The global concentration field is denoted $c$ and $h$ denotes the switching functions.
The phase fields, phase compositions and the global concentration are all dimensionless fields in the range $[0,1]$.

\section{Model}

\subsection{Free energy}
The free energy is given by
\begin{equation}
  F = f^{ch}(\vbf{\eta},\vbf{x})+f^{int}(\vbf{\eta})+f^d(\vbf{\eta},\dots) ,
  \label{eq:F}
\end{equation}
with $f^{ch}$ and $f^{int}$ representing the chemical and interface energies respectively. 
$f^d$ represents the contribution from any other physics included in the model e.g. deformation, temperature, electric field etc.

Here we take
\begin{equation}
  f^{ch}=\sum_ih_i(\vbf{\eta})f_i(x_i,\vbf{\eta}),
  \label{eq:fch}
\end{equation}
with $f_i(x_i,\vbf{\eta})=\frac{G_i(x_i)}{V^m}$ where $G_i$ is a representation of the Gibbs energy of phase $i$ and $V^m$ denotes the molar volume which is assumed to be constant.
The interface energy is taken as
\begin{equation}
  f^{int}=m\left(\sum_i\left(\frac{\eta_i^4}{4}-\frac{\eta_i^2}{2}\right)+\sum_i\sum_{j\ne i}\frac{\beta}{2}\eta_i^2\eta_j^2+\frac{1}{4}\right)+\frac{\alpha}{2}\sum_i\left(\gradient\eta_i\right)^2,
  \label{eq:fint}
\end{equation}
with $m$, $\alpha$ and $\beta$ being parameters.

The energy due to deformation is taken as 
\begin{equation}
  f^d = \frac{1}{2}\vbf{\sigma}:\vbf{\varepsilon}^e
\end{equation}
for the elastic phases (\ce{Cu} and \ce{Cu6Sn5}).
Here, $\vbf{\sigma}$ and $\vbf{\varepsilon}^e$ denotes the Cauchy stress and the elastic part of the true strain tensor respectively.
For the phases where plasticity is included (\ce{Sn}) the deformation energy is
\begin{equation}
  f^d = \frac{1}{2}\vbf{T}:\vbf{E}^e+\frac{1}{2}Q\sum_\alpha^n\sum_\beta^n q_{\alpha\beta}s^\alpha s^\beta
\end{equation}
where $\vbf{T}$ is the second Piola-Kirchoff stress tensor, $\vbf{E}^e$ is the elastic Green strain tensor, $Q$ is a hardening parameter, $q_{\alpha\beta}=1+0.4\delta_{\alpha\beta}$ controls the ratio between self ($\alpha=\beta$) and cross ($\alpha\neq\beta$) hardening, and $s^\alpha$ is the slip resistance on the $\alpha$ slip system and the sum is taken over all $n$ slip systems. \hl{It would be nicer to have the same elastic energy for all phases}
More details on the mechanical model are given in \cref{sec:crypla}.

\subsection{Governing equations}
\subsubsection{KKS conditions}
KKS models require the chemical potential to be equal in all coexisting phases, i.e.
\begin{equation}
  \partdiff{F}{c}=\partdiff{f_1}{x_1}+\partdiff{f^d_1}{x_1}=\dots=\partdiff{f_n}{x_n}+\partdiff{f^d_n}{x_n}=\mu_1\dots=\mu_n.
  \label{eq:KKS}
\end{equation}
Furthermore, mass conservation is fulfilled by
\begin{equation}
  c=\sum_ih_ix_i.
  \label{eq:mass}
\end{equation}
\subsubsection{Cahn-Hilliard equation}
The evolution of the concentration field follows the Cahn-Hilliard equation
\begin{equation}
  \partdiff{c}{t}=\gradient\cdot M(\vbf{\eta},c)\gradient \vardiff{F}{c}=\gradient\cdot M(\vbf{\eta},c)\gradient\left(\partdiff{F}{c}-\gradient\cdot\partdiff{F}{\gradient c}\right),
  \label{eq:CH1}
\end{equation}
where $M$ is the mobility.
Since there is no dependence on $\gradient c$ in the free energy the divergence term of \cref{eq:CH1} vanishes.\footnote{This is only true for KKS models, in other multiphase models a term like $\kappa\left(\gradient c\right)^2$ is included.
This term will make the Cahn-Hilliard a fourth order PDE.}

Using \cref{eq:F}, \cref{eq:fch} and \cref{eq:fint} the derivative become
\begin{equation}
  \partdiff{F}{c}=\partdiff{f^{ch}}{c}+\partdiff{f^d}{c}=\sum_i h_i\partdiff{f_i}{x_i}\partdiff{x_i}{c}+\cancelto{0}{\partdiff{h_i}{c}f_i}+\partdiff{f^d}{c}.
  \label{eq:dFdc1}
\end{equation}
By taking the derivative of the mass conservation \cref{eq:mass} we get
\begin{equation}
  \partdiff{c}{c}=\sum_i h_i\partdiff{x_i}{c}=1.
\end{equation}
Inserting into \cref{eq:dFdc1} and taking into account \cref{eq:KKS} we end up with
\begin{equation}
  \partdiff{F}{c}=\mu,
\end{equation} 
where we can arbitrarily choose $\mu=\mu_i$ based on the KKS condition.\todo{Check how $f^d$ affects this. $\partdiff{f_i}{x_i}\neq\mu_i$}
With this \cref{eq:CH1} can be written as
\begin{equation}
  \partdiff{c}{t}=\gradient\cdot M(\vbf{\eta},c)\gradient\mu_i
  \label{eq:CH}
\end{equation}
The mobility $M$ is taken as
\begin{equation}
  M = \sum_i h_i\frac{D_i}{\secdiff{f_i}{x_i}}.
\end{equation}
The units of \cref{eq:CH} are
\begin{equation}
\si[per-mode=fraction]{\per\second}=\si[per-mode=fraction]{\per\meter}\frac{\si[per-mode=fraction]{\square\meter\per\second}}{\si[per-mode=fraction]{\joule\per\cubic\meter}}\si[per-mode=fraction]{\per\meter}\si[per-mode=fraction]{\joule\per\cubic\meter}=\si[per-mode=fraction]{\per\second}.
\end{equation}


\subsubsection{Allen-Cahn equation}
The evolution of each order parameter field follow the Allen-Cahn equation
\begin{equation}
\begin{aligned}
  \partdiff{\eta_i}{t}&=-L(\vbf{\eta})\vardiff{F}{\eta_i}+\xi(\vbf{x},t)=-L\left(\partdiff{F}{\eta_i}-\gradient\cdot\partdiff{F}{\gradient\eta_i}\right)+\xi(\vbf{x},t)\\
  &=-L\left(\partdiff{f^{ch}}{\eta_i}+\partdiff{f^{int}}{\eta_i}+\partdiff{f^d}{\eta_i}-\gradient\cdot\partdiff{f^{int}}{\gradient\eta_i}-\gradient\cdot\partdiff{f^d}{\gradient\eta_i}\right)+\xi(\vbf{x},t),
\end{aligned}
\label{eq:AC1}
\end{equation}
where $\xi(\vbf{x},t)$ is a Langevin force term used to simulate nucleation, see \cref{sec:nuc}.
The derivatives of the interface energy are easy to calculate from \cref{eq:fint}
\begin{equation}
  \partdiff{f^{int}}{\eta_i}-\gradient\partdiff{f^{int}}{\gradient\eta_i}=m\left(\eta_i^3-\eta_i+2\beta\eta_i\sum_{j\ne i}\eta_j^2\right)-\gradient\cdot\alpha\gradient\eta_i.
\end{equation}

The derivative of the chemical energy is
\begin{equation}
  \partdiff{f^{ch}}{\eta_i}=\sum_j\left(\partdiff{h_j}{\eta_i}f_j+h_j\partdiff{f_j}{\eta_i}\right),
  \label{eq:dfchdeta}
\end{equation}
where
\begin{equation}
  h_j\partdiff{f_j}{\eta_i}=h_j\underbrace{\partdiff{f_j}{G_j}\partdiff{G_j}{x_j}}_{\mu_j}\partdiff{x_j}{h_j}\partdiff{h_j}{\eta_i} =\mu_jh_j\partdiff{x_j}{h_j}\partdiff{h_j}{\eta_i}.
\label{eq:dfjdetai}
\end{equation}\todo{\cref{eq:dfjdetai} will change with $f^d$}
Using that $\partdiff{c}{\eta_i}=0$, equation \cref{eq:mass} then gives
\begin{equation}
\begin{aligned}
  \partdiff{c}{\eta_i}&=\sum_j\left(\partdiff{h_j}{\eta_i}x_j+h_j\partdiff{x_j}{h_j}\partdiff{h_j}{\eta_i}\right)=\sum_j\partdiff{h_j}{\eta_i}\left(x_j+h_j\partdiff{x_j}{h_j}\right)=0 \\
  &\iff \sum_jh_j\partdiff{x_j}{h_j}\partdiff{h_j}{\eta_i}=-\sum_j\partdiff{h_j}{\eta_i}x_j.
\end{aligned}
\end{equation} 
By making use of the KKS condition \cref{eq:dfchdeta} can then be written
\begin{equation}
  \partdiff{f^{ch}}{\eta_i}=\sum_j\partdiff{h_j}{\eta_i}\left(f_j-\mu_jx_j\right)
\end{equation}\todo{will change with $f^d$}

Equation \cref{eq:AC1} now becomes
\begin{multline}
  \partdiff{\eta_i}{t}=-L\left(\sum_j\partdiff{h_j}{\eta_i}\left(f_j-\mu_jx_j\right)\right. \\
  \left.+m\left(\eta_i^3-\eta_i+2\beta\eta_i\sum_{j\ne i}\eta_j^2\right)-\gradient\cdot\alpha\gradient\eta_i+\partdiff{f^d}{\eta_i}-\gradient\cdot\partdiff{f^d}{\gradient\eta_i}\right).
  \label{eq:AC}
\end{multline}
The mobility $L$ is taken as
\begin{equation}
  L(\vbf{\eta})=\frac{\sum_i\sum_{j\ne i}L_{ij}\eta_i^2\eta_j^2}{\sum_i\sum_{j\ne i}\eta_i^2\eta_j^2}.
\end{equation} 
$L_{ij}$ is taken as 
\begin{equation}
  L_{ij}=\frac{2m}{3\alpha\left(x_i^{eq}-x_j^{eq}\right)^2}\frac{M_i+M_j}{2},
  \label{eq:lab}
\end{equation}
where $M_i=\frac{D_i}{\secdiff{f_i}{x_i}}$ and $M_j=\frac{D_j}{\secdiff{f_j}{x_j}}$ are the mobilities and $x_i^{eq}$ and $x_j^{eq}$ are the equilibrium molar fractions of phases $i$ and $j$ respectively.  

The units of \cref{eq:AC} are
\begin{equation}
\si[per-mode=fraction]{\per\second}=\frac{\si[per-mode=fraction]{\joule\per\cubic\meter}}{\si[per-mode=fraction]{\joule\per\meter}}\frac{\si[per-mode=fraction]{\square\meter\per\second}}{\si[per-mode=fraction]{\joule\per\cubic\meter}}\si[per-mode=fraction]{\joule\per\cubic\meter}=\si[per-mode=fraction]{\per\second}.
\end{equation}
\subsection{Nucleation}\label{sec:nuc}
Nucleation of the intermetallic phases is achieved using the so called Langevin force approach where a small pertubation, $\xi$, is added to the Allen-Cahn equations to initiate the evolution of the phase fields.
Following \cite{shen2007effect} the pertubation is taken as
\begin{equation}
  \xi(\vbf{x},t)=\sqrt{\frac{2k_BTL}{\lambda^d\Delta t}}\rho,
\end{equation}
where $k_B$ is the Boltzmann constant, $T$ is temperature, $\lambda$ is the grid spacing, $d$ is the dimensionality of the grid (1, 2 or 3 for 1D, 2D or 3D), $\Delta t$ is the timestep, and $\rho$ is a random number drawn from a Gaussian distribution that satisfies $\langle\rho_i\rangle=0$ and $\langle\rho_i\rho_j\rangle=\delta_{ij}$. \todo{Check what the second requirement means and if it is fulfilled by Moose}
\subsection{Non-constant molar volume}
I'm not sure it is possible to easily remove the assumption of constant molar volume.
It is certainly not as easy as I thought before.
For non-constant molar volume \cref{eq:mass} becomes $c=\sum_ih_i\frac{x_i}{V^m_i}$ which converts $c$ to units of \si{\mol\per\cubic\meter}.
The problem is that the free energy is formulated in units of \si{\joule\per\cubic\meter}, which means that $\partdiff{F}{c}$ get the units \si{\joule\per\mole}.
I don't think this is good neither for the KKS conditions nor for the Cahn-Hilliard equation.
Perhaps we can formulate the \cref{eq:CH} as
\begin{equation}
  \frac{1}{V^m}\partdiff{c}{t}=\gradient\cdot M(\vbf{\eta},c)\gradient\mu_i
\end{equation}
together with the mass balance in \cref{eq:mass} and still have $c$ dimensionless.
Then the question is how to choose $V^m$?

Another way is maybe to formulate the energies in \si{\joule\per\mol} but then I guess the mesh will represent a certain number of moles instead of a certain volume? 

\subsection{Crystal plasticity}\label{sec:crypla}
The crystal plasticity model implemented in Moose is from Zhao et al. \cite{zhao2017plastic}.
The deformation gradient $\vbf{F}$ is decomposed into an elastic part $\vbf{F}^e$ and a plastic part $\vbf{F}^p$ according to
\begin{equation}
\vbf{F}=\vbf{F}^e\vbf{F}^p.
\end{equation}
The evolution of the plastic velocity gradient is given by
\begin{equation}
  \dot{\vbf{F}}^p\vbf{F}^{p-1}=\sum_\alpha^n\dot{\gamma}^\alpha\vbf{m}^\alpha\otimes\vbf{n}^\alpha,
\end{equation}
where $\dot{\gamma}^\alpha$, $\vbf{m}^\alpha$, and $\vbf{n}^\alpha$ denotes the slip rate, slip direction, and slip plane normal on slip system $\alpha$.
The flow rule for the slip rate is 
\begin{equation}
  \dot{\gamma}^\alpha=\dot{\gamma}_0\abs{\frac{\tau^\alpha}{s^\alpha}}^{1/m}
\end{equation}

\subsection{Weak form residual equations}
The weak form of the residual equation for $\eta_i$ is constructed from \cref{eq:AC1} by multiplication with a test function $\psi$ and integrating over the domain:
\begin{equation}
\begin{aligned}
  \vbf{R}_{\eta_i}&=\vint{\partdiff{\eta_i}{t}}{\psi}+\vint{L\left(\partdiff{f^{ch}}{\eta_i}+\partdiff{f^{int}}{\eta_i}+\partdiff{f^d}{\eta_i}\right)}{\psi}-\vint{L\gradient\cdot\alpha\gradient\eta_i}{\psi}+\vint{\xi}{\psi} \\
  &=\vint{\partdiff{\eta_i}{t}}{\psi}+\vint{L\left(\partdiff{f^{ch}}{\eta_i}+\partdiff{f^{int}}{\eta_i}+\partdiff{f^d}{\eta_i}\right)}{\psi}+\vint{\alpha\gradient\eta_i}{\gradient L\psi}\\
  &+\vint{\xi}{\psi}-\sint{(\alpha\gradient\eta_i)\cdot\vbf{n}}{L\psi},
\end{aligned}
\end{equation}\todo{term with $\partdiff{f^d}{\gradient\eta_i}$ is missing}
with the divergence theorem used on the gradient term.
Inserting the derivatives calculated above gives
\begin{equation}
\begin{aligned}
  \vbf{R}_{\eta_i}=&\vint{\partdiff{\eta_i}{t}}{\psi}+\vint{L\sum_j\partdiff{h_j}{\eta_i}f_j}{\psi}-\vint{L\partdiff{h_j}{\eta_i}\mu_jx_j}{\psi}\\
  &+\vint{Lm\left(\eta_i^3-\eta_i+2\beta\eta_i\sum_{j\ne i}\eta_j^2\right)}{\psi} \\
  &+\vint{L\partdiff{f^d}{\eta_i}}{\psi}+\vint{\alpha\gradient\eta_i}{\gradient L\psi} -\sint{(\alpha\gradient\eta_i)\cdot\vbf{n}}{L\psi}+\vint{\xi}{\psi}
 \end{aligned}
\end{equation}


The Cahn-Hilliard equation \cref{eq:CH} can be solved in two ways.
Either you can solve it directly using the residual
\begin{multline}
  \vbf{R}_c=\vint{\partdiff{c}{t}}{\psi}-\vint{\gradient\cdot M\gradient\mu}{\psi}=\\ \vint{\partdiff{c}{t}}{\psi}+\vint{M\gradient\mu}{\gradient\psi}-\sint{M\gradient\mu\cdot\vbf{n}}{\psi}.
\end{multline}
This seems to be fairly straight forward, but for some reason it is ``not fully implemented'' in Moose.
Instead, the equation is split into two and solved for the concentration $c$ as well as for the chemical potential $\mu$.
The residuals for this are
\begin{align}
  &\vbf{R}_\mu=\vint{\partdiff{c}{t}}{\psi}+\vint{M\gradient\mu}{\gradient\psi}-\sint{M\gradient\mu\cdot\vbf{n}}{\psi} \\
  &\vbf{R}_c=\vint{\partdiff{F}{c}-\mu}{\psi}.
\end{align}

\section{Implementation}
This section describes the input file used to run the simulation.
Moose input files consists of different blocks which together defines the model.
All necessary blocks are described below.
\subsection{Mesh}
The mesh block can be used to create meshes of simple geometries.
It can also be used to read the mesh from a file.
A rectangular 2D mesh is created like this:
\begin{verbatim}
[Mesh]
  type = GeneratedMesh
  dim = 2
  nx = 100
  ny = 10
  xmax = 0.304 # Length 
  ymax = 0.0257 # Height
[]
\end{verbatim}
\subsection{Variables}
The \texttt{Variables} block declares the variables of the model.
A variables is defined like this:
\begin{verbatim}
[Variables]
  # concentration Sn
  [./c] # Variable name
      order = FIRST
      family = LAGRANGE
  [../]
[]
\end{verbatim}
To solve the model the following variables are needed: $c$,$\mu$,$x_i$,$\eta_i$, i.e. a simulation with three grains of different phases contains $8$ variables.
\subsection{ICs and BCs}
Initial conditions can be set in the \texttt{ICs} block. 
For simple initial conditions the type \texttt{FunctionIC} can be used.
The following code will initiate the variable $\eta_{imc}$ as a ring with radius $8$ centered at $(20,20)$.
\begin{verbatim}
[ICs]
    [./eta2] #Cu6Sn5
        variable = eta_imc
        type = FunctionIC
        function = 'r:=sqrt((x-20)^2+(y-20)^2);if(r>8&r<=16,1,0)'
    [../]
[]
\end{verbatim}
There are some more complicated initial conditions already implemented but probably I'll have to make something myself.

Boundary conditions are set in the \texttt{BCs} block using a very similar syntax as the ICs.
\subsection{Kernels}
In Moose each part of the residual equations are implemented in separate kernels. 
Each kernel is responsible for evaluating the residual and (optionally) calculate the Jacobian. 
For each of the Allen-Cahn equations the kernels in \cref{tab:AC} is needed.
For the Cahn-Hilliard equation and the KKS conditions the kernels in \cref{tab:CH} are needed.

\begin{table}
\caption{Kernels needed for the Allen-Cahn equation}
\label{tab:AC}
\begin{tabular}{lll}
  \toprule
  Residual term & Parameters &  Kernel name \\
  \midrule
  $\vint{\partdiff{\eta_i}{t}}{\psi}$ & & \texttt{TimeDerivative} \\
  $\vint{L\sum_j\partdiff{h_j}{\eta_i}f_j+w\partdiff{g}{\eta_i}}{\psi}$ & $L$, $w=0$ & \texttt{KKSMultiACBulkF}\tablefootnote{This kernel includes a double well that can be use to constrain the order parameters to $[0,1]$, turn off the double well by setting $w=0$.} \\
  $-\vint{L\partdiff{h_j}{\eta_i}\mu_jx_j}{\psi}$ & $L$ & \texttt{KKSMultiACBulkC} \\
  $\vint{L\partdiff{f^d}{\eta_i}}{\psi}$ & & \\
  $\vint{Lm\left(\eta_i^3-\eta_i+2\beta\eta_i\sum_{j\ne i}\eta_j^2\right)}{\psi}$ & $m$, $\beta$ &  \texttt{ACGrGrMulti} \\
  $\vint{\alpha\gradient\eta_i}{\gradient L\psi}$ & $L$, $\alpha$ & \texttt{ACInterface}\\
  $\vint{\xi}{\psi}$ & & \texttt{LangevinNoise} \\
  \bottomrule
\end{tabular}
\end{table}

\begin{table}
\caption{Kernels needed for the split version of the Cahn-Hilliard equation and the KKS conditions}
\label{tab:CH}
\begin{tabular}{lll}
  \toprule
  Residual term & Parameters &  Kernel name \\
  \midrule
  $\vint{\partdiff{c}{t}}{\psi}$ & & \texttt{CoupledTimeDerivative} \\
  $\vint{M\gradient\mu}{\gradient\psi}$ & $M$ & \texttt{SplitCHWRes} \\
  $\vint{\partdiff{F}{c}-\mu}{\psi}$ &  & \texttt{KKSSplitCHCRes} \\
  \midrule
  $\mu=\mu_i=\mu_j$ & & \texttt{KKSPhaseChemicalPotential} \\
  $c=\sum_ih_ix_i$ & & \texttt{KKSMultiPhaseConcentration} \\
  \bottomrule
\end{tabular}
\end{table}

\subsection{Materials}
The free energies and the switching functions are defined in the \texttt{Materials} block of the input file.
The chemical free energy of each phase can be described using the type \texttt{DerivativeParsedMaterial}.
This type uses automatic differentiation to calculate the derivatives.
\begin{verbatim}
[Materials]
  [./fch_cu] #Chemical energy Cu phase
        type = DerivativeParsedMaterial
        f_name = fch_cu
        args = 'c_cu'
        function = '20*(c_cu-0.1)^2'
  [../]
[]
\end{verbatim}

The switching functions are set like
\begin{verbatim}
  [./h_cu]
        type = SwitchingFunctionMultiPhaseMaterial
        h_name = h_cu
        all_etas = 'eta_cu eta_imc eta_sn'
        phase_etas = eta_cu
  [../]
\end{verbatim}
The same switching function can be used to describe all grain of the same phase (I think).

The materials block can also be used to define constants. For example
\begin{verbatim}
  [./constants]
      type = GenericConstantMaterial
      prop_names  = 'L kappa gamma mu tgrad_corr_mult'
      prop_values = '1. 0.5 0.5 1. 0.'
  [../]
\end{verbatim}
\subsection{Executioner, Preconditioning and Outputs}
The parameters of the solver are set in the \texttt{Executioner, Preconditioning and Outputs} blocks, here you can also set options for the behavior of the underlying PETSc commands.
The blocks can look something like this
\begin{verbatim}
[Executioner]
  type = Transient
  solve_type = 'PJFNK'
  petsc_options_iname = '-pc_type -sub_pc_type   -sub_pc_factor_shift_type'
  petsc_options_value = 'asm       ilu            nonzero'
  l_max_its = 30
  nl_max_its = 10
  l_tol = 1.0e-4
  nl_rel_tol = 1.0e-10
  nl_abs_tol = 1.0e-11

  num_steps = 100
  dt = 0.5
[]

[Preconditioning]
  active = 'full'
  [./full]
    type = SMP
    full = true
  [../]
  [./mydebug]
    type = FDP
    full = true
  [../]
[]

[Outputs]
  exodus = true
[]
\end{verbatim}
These settings will solve the equations using the Pre-Conditioned Jacobian-Free Newton-Krylow method which is the standard way in Moose.
The results will be written to a file that can be viewed in e.g. Peacock or Paraview.

\bibliographystyle{plain}
\bibliography{ref}
\end{document}